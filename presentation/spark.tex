\documentclass[xcolor=dvipsnames,compact]{beamer}
\usepackage[utf8]{inputenc}
\usepackage[czech]{babel}
\usepackage{graphicx}
\usepackage{ulem}
\usepackage{tikz}
\usetheme{Malmoe} 

\definecolor{atapurple}{HTML}{533D78}
\usecolortheme[named=atapurple]{structure} 
\setbeamertemplate{navigation symbols}{}


\definecolor{atapurple}{HTML}{533D78}
\setbeamertemplate{caption}{\raggedright\insertcaption\par}
\useoutertheme{infolines}
\useinnertheme{circles}

\makeatletter
\setbeamertemplate{title page}[default][left,colsep=-4bp,rounded=true,shadow=\beamer@themerounded@shadow]
\makeatother

\usepackage{smartdiagram}
\usepackage{tabularx}
\newcolumntype{C}{>{\centering\arraybackslash}X}
\usepackage{fancyvrb}
\newcommand\tab[1][1cm]{\hspace*{#1}}
\usepackage[default,osfigures,scale=0.95]{opensans}
\usepackage[T1]{fontenc}

\setbeamercolor{title}{fg=white}
\setbeamercolor{subtitle}{fg=white}
\setbeamercolor{author}{fg=white}
\setbeamercolor{institute}{fg=white}
\setbeamercolor{date}{fg=white}
\setbeamerfont{title}{size=\Huge}



% šablona z webu http://voho.cz

\title{Apache Spark}
\subtitle{Lightning-fast cluster computing}

%\author{\texorpdfstring{Adam Juraszek\newline\ \small{\url{adam.juraszek@ataccama.com}}}{Adam Juraszek}}
\author{Adam Juraszek}
\institute{Ataccama Software, s.r.o.}
\date{\today}

\smartdiagramset{text width=.5\textwidth,
	back arrow disabled=true,
	uniform arrow color=true,
	module y sep=1.25}

\begin{document}

{
	\usebackgroundtemplate{
		\includegraphics[height=\paperheight,trim={9cm 0 0 0},clip]{back.png}}
\begin{frame}[plain]
  \titlepage
\end{frame}
}

\begin{frame}
	\frametitle{Přehled} 
	\tableofcontents
\end{frame}

\section{Spark}

\begin{frame}
	\frametitle{Historie}
	\begin{itemize}
		\item Matei Zaharia v 2009 na Kalifornské univerzitě v Berkeley
		\item BSD 2010
		\item Apache 2013
		\item Verze 1.0 (květen 2014)
		\item[*] Rozšířená verze 1.6.3 (listopad 2016)
		\item Verze 2.0 (červenec 2016)
		\item Aktuálně verze 2.1.1 (květen 2017) \note{HDP 2.6 (duben 2017)}
	\end{itemize}
\end{frame}

\begin{frame}
	\frametitle{Vlastnosti}
	Spark je framework pro distribuované výpočty
	\begin{itemize}
		\item Rychlý
		\item Jednoduchý na použití
		\item Obecný
		\item Široce podporovaný \note{Compatible with Hadoop}
		\item Odolný k chybám
	\end{itemize}
\end{frame}

\begin{frame}
	\frametitle{Moduly a jazyky}
	\begin{columns}[T] % align columns
		\begin{column}{.48\textwidth}
			Spark se skládá z několika modulů
			\begin{itemize}
				\item[*] Core
				\item SQL and DataFrames
				\item MLlib
				\item Spark Streaming
				\item GraphX
			\end{itemize}
		\end{column}
		\begin{column}{.48\textwidth}
			Téměř shodné API pro následující jazyky
			\begin{itemize}
				\item[*] Java
				\item Scala
				\item Python
				\item R
			\end{itemize}
		\end{column}
	\end{columns}
\end{frame}

\section{Architektura}

\begin{frame}[t]
	\frametitle{Architektura}
	\begin{description}
		\item[Driver] Obsahuje uživatelský program a vytváří SparkContext
		\item[Worker] Vykonává vlastní distribuovaný výpočet
		\item[Cluster manager] Přiděluje a spravuje uzly v clusteru
		\begin{itemize}
			\item Local, Standalone, YARN, Mesos
		\end{itemize}
	\end{description}
\only<1>{
	\begin{figure}
		\begin{center}
			\includegraphics[width=.8\textwidth]{architecture.eps}
		\end{center}
	\end{figure}
}
\only<2>{
	\begin{figure}
		\begin{center}
			\includegraphics[width=.35\textwidth]{standalone.eps}
			\caption{Standalone}
		\end{center}
	\end{figure}
}
\only<3>{
	\begin{columns}[T] % align columns
		\begin{column}{.48\textwidth}
			\begin{figure}
				\begin{center}
					\includegraphics[width=\textwidth]{yarn-client.eps}
					\caption{Yarn-client}
				\end{center}
			\end{figure}
		\end{column}
		\begin{column}{.48\textwidth}
			\begin{figure}
				\begin{center}
					\includegraphics[width=0.65\textwidth]{yarn-cluster.eps}
					\caption{Yarn-cluster}
				\end{center}
			\end{figure}
		\end{column}
	\end{columns}
}
\end{frame}

\section{RDD}

\begin{frame}
	\frametitle{RDD}
	\begin{columns}[T] % align columns
		\begin{column}{.48\textwidth}
			Resilient Distributed Dataset 
			\begin{itemize}
				\item Kolekce záznamů
				\begin{itemize}
					\item Distribuovaná
					\item Neměnná
					\item Líná
					\item Typovaná
					\item Cachovatelná
				\end{itemize}
				\item Seznam oddílů
			\end{itemize}
		\end{column}
		\begin{column}{.48\textwidth}
			\begin{figure}
				\begin{center}
					\vskip -1cm
					\includegraphics[width=0.3\textwidth]{partitions.eps}
				\end{center}
			\end{figure}
		\end{column}
	\end{columns}
\end{frame}

\begin{frame}[fragile]
	\frametitle{Vznik RDD} 
	\begin{itemize}
		\item Metodou SparkContextu:
		\begin{itemize}
			\item \verb|sc.parallelize(collection)|
			\item \verb|sc.hadoopFile(path, inputFormatClass,| \\
				\tab \verb|keyClass, valueClass)|
			\item \verb|sc.textFile(path)|
			\item \verb|sc.wholeTextFiles(path)|
		\end{itemize}
		\item Transformací z existujícího RDD
	\end{itemize}
\end{frame}

\begin{frame}[fragile]
	\frametitle{Transformace RDD}
	\begin{itemize}
		\item Odvození RDD z jednoho nebo více rodičů
		\item Líné vyhodnocení
		\item Příklady:
		\begin{itemize}
			\item \verb|rdd.filter(f)|
			\item \verb|rdd.map(f)|, \verb|rdd.flatMap(f)|, \verb|rdd.mapPartitions(f)|
			\item \verb|rdd.intersection(rdd2)|, \verb|rdd.union(rdd2)|
			\item \verb|rdd.cartesian(rdd2)|
			\item \verb|rdd.distinct()|
			\item \verb|rdd.coalesce(n, shuffle)|, \verb|rdd.repartition(n)|
		\end{itemize}
		\item Další transformace pro párové RDD:
		\begin{itemize}
			\item \verb|rdd.groupByKey()|
			\item \verb|rdd.reduceByKey(f)|
			\item \verb|rdd.join(rdd2)|
			\item \verb|rdd.cogroup(rdd2)|
		\end{itemize}
		\item Transformace může vést k vytvoření více než jednoho RDD
	\end{itemize}
\end{frame}

\begin{frame}[fragile]
	\frametitle{Akce na RDD} 
	\begin{itemize}
		\item Spouští výpočet
		\item Vrací data na Driver
		\item Příklady:
		\begin{itemize}
			\item \verb|rdd.reduce(f)|, \verb|rdd.count()|, \verb|rdd.first()|
			\item \verb|rdd.collect()|, \verb|rdd.take(n)|
			\item \verb|rdd.saveAsTextFile(path)|, \verb|rdd.saveAsNewAPIHadoopFile(path, | \\
				\tab \verb|keyClass, valueClass, outputFormatClass)|
			\item \verb|rdd.foreach(f)|
		\end{itemize}
	\end{itemize}
\end{frame}

\begin{frame}[fragile]
	\frametitle{Implementace RDD}
	\begin{itemize}
		\item RDD je abstraktní třída:
		\begin{itemize}
			\item Vstupní: CheckpointRDD, NewHadoopRDD, ParallelCollectionRDD, WholeTextFileRDD
			\item Úzké: CartesianRDD, MapPartitionsRDD, UnionRDD, ZippedPartitionsRDD
			\item Široké: CoalescedRDD, CoGroupedRDD, ShuffledRDD, SubtractedRDD
		\end{itemize}
		\item Každý potomek definuje:
		\begin{itemize}
			\item \verb|def compute(Partition, TaskContext): Iterator[T]|
			\item \verb|def getPartitions: Array[Partition]|
			\item \verb|def getDependencies: Seq[Dependency[_]]|
			\item \verb|def getPreferredLocations(Partition): Seq[String]|
			\item \verb|val partitioner: Option[Partitioner]|
		\end{itemize}
		\item Další vlastnosti RDD:
		\begin{itemize}
			\item \verb|val id: Int|
			\item \verb|var name: String|
			\item \verb|var storageLevel: StorageLevel|
			\item \verb|var checkpointData: Option[RDDCheckpointData[T]]|
		\end{itemize}
	\end{itemize}
\end{frame}

\section{Plány a spouštění}

\begin{frame}
	\frametitle{Výpočet} 
	\begin{center}
		\smartdiagram[flow diagram]{
			Zavolání akce,
			Logický plán se převede na fyzický plán,
			Naplánují se úlohy ve fázích se splněnými závislostmi,
			Po dokončení všech fází je výsledek zaslán driveru,
			Je proveden checkpoint
		}
	\end{center}
\end{frame}

\begin{frame}
	\frametitle{Logický plán}
	\begin{itemize}
		\item Zachycuje závislosti RDD na předcích
		\item Formálně: orientovaný acyklický graf, kde vrcholy jsou RDD a hrany jsou transformace
	\end{itemize}
	\begin{figure}
		\begin{center}
			\includegraphics[width=.75\textwidth]{word-count.eps}
		\end{center}
	\end{figure}
\end{frame}

\begin{frame}[t]
	\frametitle{Závislosti v logickém plánu}
	Typy závislostí jsou součástí logického plánu.
	\begin{description}
		\item[Úzká] Oddíl potomka závisí na malém množství celých oddílů potomka.
		Často platí: Na oddílu předka závisí nejvýše jeden oddíl potomka.
		\only<1>{
		\begin{columns}[T] % align columns
			\begin{column}{.33\textwidth}
				\begin{figure}
					\begin{center}
						\includegraphics[width=0.9\textwidth]{unions.eps}
						\SaveVerb{x}|rdd1.union(rdd2)|
						\caption{\protect\UseVerb{x}}
					\end{center}
				\end{figure}
			\end{column}
			\begin{column}{.33\textwidth}
				\begin{figure}
					\begin{center}
						\includegraphics[width=0.9\textwidth]{coalesce.eps}
						\SaveVerb{x}|rdd.coalesce(2, false)|
						\caption{\protect\UseVerb{x}}
					\end{center}
				\end{figure}
			\end{column}
			\begin{column}{.33\textwidth}
				\begin{figure}
					\begin{center}
						\includegraphics[width=0.9\textwidth]{cartesian.eps}
						\SaveVerb{x}|rdd1.cartesian(rdd2)|
						\caption{\protect\UseVerb{x}}
					\end{center}
				\end{figure}
			\end{column}
		\end{columns}
		}
		\item[Široká]<2> Na oddílu předka závisí více oddílů potomka.
		Potomci závisí jen na části oddílu předka.
		\begin{columns}[T] % align columns
			\begin{column}{.33\textwidth}
				\begin{figure}
					\begin{center}
						\includegraphics[width=0.9\textwidth]{shuffled.eps}
						\SaveVerb{x}|rdd.coalesce(2, true)|
						\caption{\protect\UseVerb{x}}
					\end{center}
				\end{figure}
			\end{column}
		\end{columns}
	\end{description}
\end{frame}

\begin{frame}
	\frametitle{Fáze}
	Logický plán je rozdělen na fáze; hranicí fáze je široká závislost.
	\begin{figure}
		\begin{center}
			\includegraphics[width=0.9\textwidth]<1>{big1.eps}
			\includegraphics[width=0.9\textwidth]<2>{big.eps}
		\end{center}
	\end{figure}
\end{frame}

\begin{frame}
	\frametitle{Úlohy} 
	\begin{itemize}
		\item Závislost fází je daná závislostmi obsažených RDD
		\item Výpočet fáze až po dokončení všech předků
		\item Každému oddílu v cílovém RDD odpovídá jedna úloha:
		\begin{description}
			\item[Výsledná úloha] -- pokud na RDD byla zavolána akce
			\item[Shuffle mapovací úloha] -- pokud následuje široká závislost
		\end{description}
	\end{itemize}
\end{frame}

\begin{frame}
	\frametitle{Výsledná úloha}
	\begin{itemize}
		\item Akce definuje:
		\begin{itemize}
			\item Úlohy, které budou potřeba
			\item Operace provedené na závěr v úloze
			\item Operace provedené na driveru po obdržení dat
		\end{itemize}
	\end{itemize}
	\begin{table}
		\begin{center}
			\begin{tabularx}{\linewidth}{ C | C | C }
				Akce & V úloze & Na driveru \\
				\hline
				count & count & sum \\
				max & max & max \\
				take & take & append \\
				top & heap-add & heap-merge \\ 
				collect & -- & append
			\end{tabularx}
		\end{center}
		%		\caption{Ukázka tabulky}
		%		\label{tab:tabulka}
	\end{table}
\end{frame}

\begin{frame}
	\frametitle{Aplikace, joby, fáze a úlohy}
	\begin{center}
		\smartdiagram[flow diagram]{
			SparkContextu odpovídá jedna aplikace,
			Akce na RDD spouští jeden nebo více jobů,
			Job se skládá z jedné nebo více fází,
			Fáze obsahuje úlohu pro každý oddíl finálního RDD
		}
	\end{center}
\end{frame}

\section{Shuffle}

\begin{frame}
	\frametitle{Shuffle}	
	Na hranici fází se provádí shuffle:
	\begin{itemize}
		\item Vždy s páry klíč -- hodnota
		\item Roztřízení výstupu podle klíče (shuffle mapovací úloha)
		\item Distribuce na vstup následující úlohy (pull)
	\end{itemize}
\end{frame}

\begin{frame}
	\frametitle{Zápis shuffle}
	\begin{description}
		\item[Segment] část výstupu určená jednomu reduceru.
	\end{description}

	\begin{center}
		\smartdiagram[flow diagram]{
			Seřaď záznamy v paměti podle klíče a udržuj hranice,
			Odlej na disk{,} pokud se nevejdou,
			Dotřiď na disku sléváním{,} komprimuj,
			Zaregistruj výstupní segmenty na driveru
		}
	\end{center}

	Do verze 1.1 byl používaný hash namísto řazení
\end{frame}

\begin{frame}[fragile]
	\frametitle{Čtení shuffle}
	
	\begin{center}
		\smartdiagram[flow diagram]{
			Dotaž se driveru na umístění segmentů předchozí fáze,
			Dotáhni paralelně potřebné segmenty,
			Data okamžitě zpracuj -- udržují se v (External)AppendOnlyMap,
			Kombinování hodnot stejného klíče je parametrizované funkcí
		}
	\end{center}

	Kombinace: \verb|map.put(key, func(value, map.get(key)))|
\end{frame}

\begin{frame}[fragile]
	\frametitle{Transformace vyžadující shuffle}
	\begin{itemize}
		\item \verb|reduceByKey|
		\item \verb|groupByKey|: funkce ve čtení shuffle je append
		\item \verb|distinct|:
		\begin{enumerate}
			\item před: \verb|map(x -> (x, null))|
			\item \verb|reduceByKey| s funkcí: \verb|(x, y) -> x|
			\item po: \verb|map((x, y) -> x)|
		\end{enumerate}
		\item \verb|cogroup|: podobné \verb|groupByKey|, hodnoty jsou v seznamech podle indexu vstupního RDD
		\item \verb|join|: \verb|cogroup| a následně \verb|flatMap|
		\item \verb|intersection|:
		\begin{enumerate}
			\item před: \verb|map(x -> (x, null))|
			\item \verb|cogroup|
			\item po: \verb|filter| a \verb|map| jako u \verb|distinct|
		\end{enumerate}
		\item \verb|repartition|: prostý append ve čtení shuffle
	\end{itemize}
\end{frame}

\section{Ukázka}

{
\begin{frame}
	\begin{tikzpicture}[remember picture,overlay]
		\node at (current page.center) {
			\includegraphics[width=\textwidth]{big.eps}
		};
	\end{tikzpicture}
	
	
	\setbeamercolor{coloredboxstuff}{fg=white,bg=atapurple}
	\begin{beamercolorbox}[wd=\textwidth,sep=1em]{coloredboxstuff}
		\begin{center}
			\Huge Následuje ukázka
		\end{center}
	\end{beamercolorbox}

\end{frame}
}

\end{document}